%**start of header
\documentclass[authoryear,round,12pt]{article}
\usepackage{pslatex}
\usepackage{color}
\usepackage{natbib}
\usepackage[pdftex]{graphicx}
\usepackage{setspace}
\usepackage{amsmath}
\renewcommand{\baselinestretch}{1.0}
\setlength\headheight{0in}
\setlength{\oddsidemargin}{-0.15in}
\setlength{\evensidemargin}{-0.15in}
\setlength{\textwidth}{6.8in}
\setlength{\textheight}{9.0in}
\setlength{\topmargin}{-0.5in}
\def\degree{\hbox{$^\circ$}}
\def\elnino{El Ni{\~n}o}
\def\lanina{La Ni{\~n}a}
\DeclareGraphicsRule{.png}{eps}{.png.bb}{`convert #1 'eps:-'}
\DeclareGraphicsRule{.jpg}{eps}{.jpg.bb}{`convert #1 'eps:-'}
\DeclareGraphicsRule{.gif}{eps}{.gif.bb}{`convert #1 'eps:-'}
\DeclareGraphicsRule{.tif}{eps}{.tif.bb}{`convert #1 'eps:-'}

% extra packages for weird stuff
%\usepackage[]{graphicx}
\usepackage[capbesideposition={bottom},facing=yes,capbesidesep=quad]{floatrow}
\usepackage{float}
\usepackage[normal]{caption}
\usepackage[version=3]{mhchem}
\usepackage[table]{xcolor}

\begin{document}
%**end of header



\sloppy
\begin{center}
 \textit{Summary page}

 \bigskip
 
 \textbf{\Large Assessing the Role of Seafloor Weathering in Carbon Cycling and Climate}

 \bigskip
Navah Farahat, David E. Archer, and Dorian S. Abbot
\end{center}

%The Project Summary should be written in the third person, informative to other persons working in the same or related fields, and, insofar as possible, understandable to a scientifically or technically literate lay reader. It should not be an abstract of the proposal.

\textbf{Overview:} The global carbon cycle 
determines the distribution of carbon between the atmosphere, ocean,
and solid earth. Subtle adjustments in this balance are believed to
maintain habitable conditions
throughout Earth's history. Low-temperature alteration of seafloor
basalt (seafloor weathering) contributes to the global carbon cycle by
provoking the precipitation of carbonates, either directly within the crust or resulting from alkalinity fluxes to the ocean.  
It has so far not
been possible to determine whether changes in seafloor weathering have
contributed to the negative climate-weathering feedback that ensures
Earth's climate stability because the necessary modeling framework has
not yet been developed. \textit{The focus of the proposed research is
  to rectify this situation by developing, testing, and using a
  comprehensive mechanistic seafloor weathering model.}


The specific main objectives of this proposal are as follows:
\begin{enumerate}
\item To develop, test, and benchmark the first comprehensive 2D
  reactive-transport model of the fluid flow and geochemical reactions
  associated with seafloor weathering.
\item To use the model to determine the effect of geological and
  climatic factors on seafloor weathering, which will constrain the
  ability of seafloor weathering to participate in long-term climate
  regulation and the evolution of a habitable planet.
\item To apply the model to the Archean carbon cycle and make
  inferences about the effect of seafloor weathering on the Faint
  Young Sun Problem.
\end{enumerate}


This proposal \textit{is within the scope of Marine Geology and
  Geophysics} because it addresses ``tectonic evolution of the
mid-ocean ridges'' and ``the processes controlling exchange of heat
and chemical species between seawater and ocean rocks.''

%The statement on intellectual merit should describe the potential of the proposed activity to advance knowledge.
\textbf{Intellectual merit:} Developing a spatially resolved model of
seafloor weathering will allow significant advances in our
understanding of the global carbon cycle. For example, it will allow
us to determine whether the carbon uptake rate is modulated by the
geothermal heat flux, the ocean bottom water temperature, or the ocean dissolved carbon concentration, and thereby
assess whether a longterm stabilizing climate-seafloor weathering
feedback can exist. This work is critical to scientists studying Earth
history (e.g., the Faint Young Sun problem), those interested in the
origin and development of life on Earth and other planets, and to
astronomers interested in which of the newly discovered exoplanets may
be in the habitable zone.



%The statement on broader impacts should describe the potential of the proposed activity to benefit society and contribute to the achievement of specific, desired societal outcomes.
\textbf{Broader impacts:} The basic research into seafloor weathering
proposed here brings us closer to a comprehensive picture of the
global carbon cycle, a framework necessary for understanding the
future fate and impact of industrial CO$_2$ on our environment.  The model
we propose to develop has other potential applications, such as
studies of deliberate or natural carbon sequestration and nuclear waste disposal in stable
geological environments, that emphasize the broader impact of this
work. Additionally, the main purpose of this proposal is to fund the
remainder of the PhD studies of Navah Farahat, who as a woman is 
an under-represented group in a
mathematical science. Finally, this proposal contains an
\textit{innovative outreach program} including broad dissemination of
educational material via online and print zines and a web-based GUI
based on a toy version of the model.



\newpage

\setcounter{page}{1}

\begin{center}
 \textbf{\Large Assessing the Role of Seafloor Weathering in Carbon Cycling and Climate }

 \bigskip
 
Navah Farahat, David E. Archer, and Dorian S. Abbot
\end{center}

\bigskip

\tableofcontents

\bigskip

\section{Results from previous NSF support}
\label{sec:prior-support}

\textit{Dorian Abbot:} Member of the Mathematics and Climate Research
Network as co-Principle Investigator on NSF
DMS-0940261. 10/01/2010-09/20/2015, \$278,260.00. This grant partially
supported three graduate students, including Navah Farahat as she
started her PhD work. It helped support a project to compare all the
GCMs that had been run for the Snowball Earth in a consistent
configuration, which led to the important recognition that clouds
should provide a significant positive radiative forcing in a Snowball
Earth
\citep{abbot12-snowball-clouds,abbot13-snowball-circulation}. The
grant also supported graduate student Daniel Koll as he worked on a
paper disentangling how changes in clouds can keep the tropical
temperature from changing much when the ocean heat transport to higher
latitudes changes \citep{Koll:2013}. Finally, it helped support
graduate student David Plotkin, as he analyzed oceanographic data
using Diffusion Maps and Spectral clustering, a new statistical
technique to detect bimodality in datasets, to demonstrate that the
Kuroshio current off the coast of Japan is bimodal
\citep{Plotkin2014}.

  
\textit{David Archer:} 

\section{Introduction}
\label{sec:introduction}
  


Carbon dioxide is an important greenhouse gas that helps keep Earth's
climate warm enough for life to thrive
\citep{Pierrehumbert:2010-book}. Carbon is exchanged between Earth's
mantle, oceans, atmosphere, and oceanic crust on geologic timescales
(Figure \ref{fig:global}). Carbon from the mantle enters Earth's
surficial environment as CO$_2$ by volcanic outgassing, and carbon is
buried in the oceanic crust as carbonate rocks during dissolution and
alteration of silicate rocks. The subduction of carbonate-rich oceanic
plates closes the cycle and returns carbon to the mantle. When the
burial rate of atmosphere-ocean carbon is controlled by climate, a
climate-weathering feedback occurs. Weathering of continental silicate
rocks, widely held to consume atmospheric CO$_2$ at a rate controlled
by temperature and precipitation, has been proposed as a long-term
climate regulation mechanism \citep{walker1981, berner1983,
  caldeira1995, berner2001, berner2006, arvidson2006}.



\begin{figure}[!hb]
\begin{center}
  \includegraphics[width=11cm]{Figs/global0.eps}
\end{center}
\caption{Cartoon of carbon exchange between
      Earth's mantle, oceans, atmosphere, and oceanic crust. The
      atmosphere and ocean carbon reservoirs equilibrate on short
      timescales compared to the inorganic carbon cycle, thus the
      atmosphere-ocean system is treated as a single reservoir in this
      illustration. Carbon fluxes in and out of the mantle are within
      ranges calculated by \citet{gerlach1989} and \citet{sano1996},
      chosen to illustrate a closed cycle. The total carbon burial
      flux is within ranges suggested by \citet{alt1999} and
      \citet{staudigel1989}, although the partitioning of 
      carbon burial between seafloor versus continental weathering is
      poorly constrained.  
%Understanding the relative importance of      seafloor weathering is one of the focuses of this      project.
}
\label{fig:global}
\end{figure}

%\begin{figure}[h]
%\thisfloatsetup{capbesideposition={center,inside},capbesidewidth=8cm}
%\fcapside[\FBwidth]
% {\caption{Cartoon of carbon exchange between
%      Earth's mantle, oceans, atmosphere, and oceanic crust. The
%      atmosphere and ocean carbon reservoirs equilibrate on short
%      timescales compared to the inorganic carbon cycle, thus the
%      atmosphere-ocean system is treated as a single reservoir in this
%      illustration. Carbon fluxes in and out of the mantle are within
%      ranges calculated by \citet{gerlach1989} and \citet{sano1996},
%      chosen to illustrate a closed cycle. The total carbon burial
%      flux is within ranges suggested by \citet{alt1999} and
%      \citet{staudigel1989}, although the partitioning of 
%      carbon burial between seafloor versus continental weathering is
%      poorly constrained.}
% \label{fig:global}}
%  {\includegraphics[width=9cm]{Figs/global0.eps}}
%\end{figure}



Seafloor weathering provides a significant pathway for transferring carbon from
the atmosphere-ocean system into the solid Earth, which involves the
low-temperature alteration of oceanic crust and the production of
carbonates formed by seawater carbonate and divalent ions leached from
the basalt. Seafloor basalts become progressively chemically altered
from the time of their creation at mid-ocean ridges until they reach
the ``sealing age,'' the age of oceanic crust when heat flux and/or
pore space are too low for circulation to expose the rock
to an environment that favors alteration \citep{stein1994}. In this alteration 
environment, geologic CO$_2$ uptake by seafloor weathering involves the 
leaching of calcium, magnesium, and iron oxides from primary basaltic minerals 
and glass into seawater-derived hydrothermal fluids, and the precipitation of 
carbonate rocks including calcite/aragonite (CaCO$_3$), dolomite 
(CaMg(CO$_3$)$_2$), and siderite (FeCO$_3$). As a result, the
ocean crust is a carbon sink with respect to the ocean-atmosphere
system, with a carbon storage rate of 1.5 x 10$^{11}$ -- 3.7 x
10$^{12}$ mol C yr$^{-1}$ \citep{alt1999, staudigel1989}. This carbon
uptake rate by oceanic crust is comparable to the rate of of CO$_2$ outgassing rates
from mid-ocean ridges \citep[1.0 -- 1.9 x 10$^{12}$ CO$_2$
yr$^{-1}$,][]{gerlach1989}.


Like continental weathering, seafloor weathering has been proposed to
consume CO$_2$ at a rate controlled in part by climate, but the link
is presently poorly understood, having received considerably less attention
\citep{francois1992, spivack1994, caldeira1995, brady1997, sleep2001,
  arvidson2006, coogan2013, arvidson2013}. A climate feedback due to seafloor
weathering would be particularly important during snowball states when less
continental surface is exposed to weathering \citep{lehir2008}, during
the Archean when seafloor spreading rates and oceanic crust production
were higher \citep{sleep2001}, and on waterworld planets where
continental weathering cannot regulate climate
\citep{abbot2012}. There are geological observations that suggest the rate of carbon 
uptake by seafloor weathering has changed in the past, possibly in response to climate. 
Upper oceanic crust from the warm Late Mesozoic has been found to have several times 
higher CO$_2$ concentration than upper oceanic crust from the Cenozoic, which had 
typically cooler climates \citep{gillis2011, coogan2013}. Though these findings suggest the carbon uptake rate by seafloor weathering was higher during warmer climates, the mechanism of the feedback is unclear. 

Previous experiments have demonstrated that basalt
alteration rates are temperature-dependent \citep{gislason1993,
  oelkers2001, gislason2003, dessert2003}, but it is not clear whether
the temperature where the relevant reactions occur is set by the deep
ocean temperature (which might produce a climate feedback) or
by the physics of fluid flow in a cooling column. The extent of basalt dissolution may 
also be limited by availability of dissolved inorganic carbon, which controls the pH 
of the solution.  If the ocean carbon inventory scales in some complex way with that
 of the atmosphere, another form of climate feedback might result.  
As another potential chemical feedback, enhanced river flux of potassium to the oceans
 during warm climates has been suggested to enhance potassium feldspar formation,
resulting in higher alkalinity flux to the ocean, ultimately driving
carbonate precipitation \citep{coogan2013}.

Hydrothermal circulation, the flow of seawater through oceanic crust, 
is clearly observed as the large discrepancy between predicted basement 
heat flow by conduction and observed ocean heat flow \citep{stein1994}, as 
hydrothermal circulation contributes significantly to the transport of heat 
from the Earth's interior to the ocean \citep{lister1980, sclater1980}. 
Near mid-ocean ridges, temperature gradients are large and circulation
through high-temperature venting is responsible for the advection of
heat. However, carbonate precipitation is not favored at high temperatures, and 
the main impact of chemical alteration at these temperatures is the precipitation of 
gypsum, resulting in acidification of the hydrothermal fluid, precluding the seafloor 
weathering reaction. Instead, the seafloor weathering reactions responsible for 
geologic CO$_2$ uptake occur throughout the permeable ocean basement of low-
temperature mid-ocean ridge flanks, known as the crustal aquifer, where hydrothermal 
circulation brings seawater-derived hydrothermal fluid into contact with oceanic basalt. 
This flow controls the temperature of the fluids responsible for seafloor 
weathering, the availability of seawater carbonate to the oceanic basalt, and 
the deposition of secondary hydrothermal minerals including but not limited to 
nontronite, saponite, stilbite and quartz (based on the Phreeqc thermodynamic core 
which will be used in our model). We are constructing a two-dimensional spatially 
resolved model of the hydrological setting of seafloor weathering based on models  
outcrop-to-outcrop hydrothermal flow in a poorly mixed aquifer \citep{fisher2000, 
anderson2012, anderson2013}. This model is a valuable tool for simulating the mixing of hydrothermal fluids and CO$_2$-rich seawater as it is sensitive to topographic 
anomalies, permeability, crustal heat flow, and sediment cover \citep{wang1996, 
miller2004}.

\begin{figure}[!hb]
\begin{center}
\includegraphics[scale=0.26]{Figs/modelUpdate.eps}
\end{center}
\caption{Illustration of two-dimensional seafloor weathering model
  with coupled fluid circulation model (blue), geochemical equilibrium
  model (green), and structural evolution model (red). Zoom view shows
  processes happening in each grid cell. \textbf{Fluid circulation:}
  At each time step, the circulation model solves for the velocity
  field of hydrothermal fluid which advects the geothermal heat, q, to
  the ocean bottom at the top of the domain. \textbf{Geochemical
    equilibrium:} At each time step, dissolved species are transported
  by advective fluxes and diffusive fluxes and a geochemical
  equilibrium sub-model \citep[PHREEQC,][]{phreeqc2013} calculates the
  equilibrium fluid composition and mineral assemblage for each grid
  cell. \textbf{Crustal evolution:} Initial porosities and
  permeabilities represent a layer of low-permeability sediment, a
  high-permeability crustal aquifer layer, and a low-permeability
  layer of sheeted dikes or gabbros. At each time step, each grid
  cell's porosity, $\phi$, and permeability, K, are updated to reflect
  volume changes in the mineral assemblage.}
\label{fig:model}
\end{figure}


\textit{The overarching goal of this proposal is to develop and use a 
two-dimensional numerical model of low-temperature basalt alteration 
and carbonate precipitation in the permeable upper oceanic crust as a 
tool to explore seafloor weathering as a climate feedback.} 
Figure \ref{fig:model} shows an illustrated
representation of the processes that will be included in the model,
including descriptions of how they are coupled. Previous studies of
seafloor weathering have made significant contributions using
qualitative, generally one-box, models \citep{caldeira1995, sleep2001,
  lehir2008, coogan2013}, and the logical next step is to extend this
work using a spatially and mechanistically resolved model. Critical questions, such as
whether geothermal processes or deep temperature set the
reaction temperature, can only be resolved with a two-dimensional
model. Moreover, recent advances in porous media flow modeling and
reactive transport modeling based on work for other geophysical and
industrial applications make this project timely.  In combination with
these modeling advances, computational power is inexpensive enough
that hydrothermal circulation can be simulated at high spatial and
temporal resolution in two dimensions. Additionally, parallel
computing allows spatially decoupled processes, such as chemical
equilibrium simulations, to run simultaneously, so spatially resolved
seafloor weathering geochemistry can be efficiently simulated
numerically.

This proposal \textit{is within the scope of Marine Geology and
  Geophysics} because it addresses ``tectonic evolution of the
mid-ocean ridges'' and ``the processes controlling exchange of heat
and chemical species between seawater and ocean rocks.''

In the following sections, we present the objectives and significance
of the proposed work (section \ref{sec:object-expect-signif}), outline
a detailed work plan including some preliminary results (section
\ref{sec:detailed-work-plan}), and discuss the coordination between
principal investigators on this project (section
\ref{sec:proj-management}).




\section{Objectives and significance of research plan}
\label{sec:object-expect-signif}


We outline the main \textit{objectives} of the proposed research,
the \textit{practical steps} we will take to achieve these objectives,
and the objectives' \textit{scientific and social significance} in
this section.

\begin{enumerate}

\item \textbf{Objective:} To develop, test, and benchmark the first
  comprehensive 2D reactive-transport model of the fluid flow and
  geochemical reactions associated with low-temperature alteration of
  basalt (seafloor weathering).

  \textbf{Significance:} Our understanding of the role of seafloor
  weathering in the global carbon cycle has been inhibited by the lack
  of a spatially resolved model. For example, it is clear that the zone of the crust where carbon-uptake reactions
  occur is limited by thermodynamics, and it may also be affected by 
changes in ocean or mantle temperature, spreading geophysics, or ocean chemistry. 
Our model will resolve the mechanisms that control carbon precipitation sufficiently to allow us to understand these 
feedback relationships, and the role of seafloor weathering in the global carbon cycle.  
The model, and the mechanistic understanding that we hope to produce from it, will also be applicable to 
carbon sequestration and nuclear waste disposal studies.

  \textbf{Practical Steps:} We have already developed and benchmarked
  the fluid dynamical and geochemical segments of the code. The next
  step is to couple these components together to form a spatially
  resolved 2D reactive transport model of seafloor weathering by coupling every grid point to the 
phreeqc aqueous chemistry thermodynamic solver package. After
  this, we will benchmark the coupled code to ensure accurate
  performance. Finally, we will parallelize the code to make it as
  numerically efficient as possible.

\item \textbf{Objective:} To use the model to determine the effect of
  factors such as bottom water temperature, pH, total dissolved
  inorganic carbon, and sediment cover on seafloor weathering, which will
  constrain the ability of seafloor weathering to participate in
  long-term climate regulation and the evolution of a habitable
  planet.

  \textbf{Significance:} These simulations will inform the question
  of to what extent seafloor weathering has participated in the
  longterm climate stabilization of planet Earth. This work is
  critical to scientists studying Earth history, those interested in
  the origin and development of life on Earth and other planets, and
  to astronomers interested in which of the newly discovered
  exoplanets may be in the habitable zone.

  \textbf{Practical Steps:} We will address this objective through an
  extensive series of carefully chosen simulations using the model we
  will develop.

\item \textbf{Objective:} To apply the model to the Archaean carbon
  cycle and make inferences about the effect of seafloor weathering on
  the Faint Young Sun Problem.

  \textbf{Significance:} The Faint Young Sun Problem remains one of
  the major outstanding problems in Earth Science. At issue is whether
  a clement climate was maintained on Earth when the insolation was
  30--40\% lower through an increase in CO$_2$ alone, or whether other
  greenhouse gases also had to have higher concentrations. 

  \textbf{Practical Steps:} We will run the model with increased
  crustal heat flux, dissolved inorganic carbon, with and without sediment cover, and using a variety of
  basalt compositions relevant for Archean Earth and determine whether
  seafloor weathering can be increased enough to limit CO$_2$ below
  the level that would be necessary to resolve the Faint Young Sun
  Problem. 



\end{enumerate}


\begin{table}[!h]
\caption{Time table for the proposed project.}\label{table:time-table} 
\begin{center}
  \begin{tabular}{lll}
 \hline
 Activity                        & Section                               & Period  \\
  \hline
Developing a comprehensive 2D model of seafloor weathering    & \ref{sec:developing}                 & Year 1  \\
Constraining the effect of seafloor weathering on climate regulation & \ref{sec:constraining}  & Year 2 \\
Applying the model to the Archaean carbon cycle & \ref{sec:paleoclimate} & Year 3 
\end{tabular}
\end{center}
\end{table}


\section{Detailed work plan}
\label{sec:detailed-work-plan}

The time table for our planned effort is summarized in Table
\ref{table:time-table}.  


\subsection{Developing a comprehensive 2D model of seafloor weathering}
\label{sec:developing}

\subsubsection{Hydrothermal circulation model}
\label{sec:hydrothermal}


In low-temperature hydrothermal circulation the temperatures are too
low for gaseous venting, so the only phases to consider are a mobile
fluid phase and an immobile rock phase. It is therefore reasonable to
assume a fluid-saturated porous medium, and use the Darcy flow
equation \citep{whitaker1985,patankar1980,durran1998,saied2004,
  saied2006}. Our implementation couples the Darcy flow equation to
equations for the conservation of heat and mass. In addition to
low-temperature hydrothermal field models, these techniques have been
used for other porous media flow applications including closed cavity
studies \citep{saied2004, saied2006} and high-temperature hydrothermal
vent studies \citep{wilcock1998, fontaine2007}.

As a demonstration that our implementation is successful, we benchmarked
it to a standard study of natural convection in a two-dimensional
square cavity of uniform porous material by \citet{saied2006}. Figure
\ref{fig:bplot1} shows an example comparison of our code with the
standard code, which demonstrates that our code can successfully
simulate hydrothermal circulation, including the difficult boundary
layers on the right and left boundaries. We have also benchmarked our
model against the two-dimensional, low-temperature hydrothermal
circulation study by \citet{snelgrove1996}. \citet{snelgrove1996}
modeled circulation through a layer of basalt that was effectively
``trapped'' by a layer of low permeability sediment, as observed on
the flanks of the Juan De Fuca ridge. We were able to reproduce their
results (not shown), further demonstrating that our hydrothermal
circulation implementation is successful.

%\begin{figure}[h!]
%\begin{center}
%\includegraphics[scale=0.3]{Figs/comp10.pdf}
%\end{center}
%\caption{Streamlines (left) and isotherms (right) at steady state in a
%  square porous cavity benchmark as described by \citet{saied2006},
%  with a Reynoldys number of 1000. Top plots are our results with 101 x 101 grid
%  resolution; bottom plots are from \citet{saied2006}.}
%\label{fig:bplot1}
%\end{figure}

\begin{figure}[h]
\thisfloatsetup{capbesideposition={center,inside},capbesidewidth=8cm}
\fcapside[\FBwidth]
 {\caption{Streamlines (left) and isotherms (right) at steady state in a
  square porous cavity benchmark as described by \citet{saied2006},
  with a Reynoldys number of 1000. Top plots are our results with 101 x 101 grid
  resolution; bottom plots are from \citet{saied2006}.}
 \label{fig:bplot1}}
  {\includegraphics[width=9cm]{Figs/comp10.pdf}}
\end{figure}

%\begin{table}[h]
%\caption{Benchmark values of $\overline{N_u}(x=0)$ evaluated at steady state from the presented model and other groups.} 
%\centering
%\begin{tabular}{c rrrrrrr} 
%\hline\hline 
%Group & $R_a=1000$ & $R_a=100$ \\ [0.5ex]   
%\hline  
%
%This study & 14.78 & 3.358\\
%Saied 2006 & 13.726 & 3.002\\
%Walker and Homsey & 12.960 & 3.097 \\
%Bejan & 15.800 & 4.200 \\
%Gross et al. & 13.448 & 3.141 \\
%Manole and Lage & 13.637 & 3.118\\
%Baytas & 13.060 & 3.160\\
%
%
%\hline      
%\end{tabular}
%\label{tab:nuss}
%\end{table}

\subsubsection{Geochemical model}
\label{sec:geochemical}



We use the United States Geological Survey low-temperature aqueous
geochemistry code PHREEQC \citep{phreeqc2013} to model low-temperature
basalt alteration and carbonate precipitation. PHREEQC is capable of
simulating interactions between mineral assemblages, dissolved gases,
and aqueous solutions in fully saturated systems. For this
application, we run individual PHREEQC equilibrium simulations at
each grid cell at each time step in the model. Mobile phase (fluid)
composition is given by advective and diffusive reactive
transport  and immobile phase (mineral
assemblage) composition is taken from the previous time step. Rate
laws describing dissolution of primary basaltic minerals and glass
also influence the fluid and mineral chemistry at each time
step\citep{gislason1993, gislason2003, pham2012}. PHREEQC is interfaced into the seafloor weathering model using
the fortran module iPhreeqc \citep{iphreeqc2011}. This allows PHREEQC to solve
for geochemical equilibrium according to conditions determined by the 
hydrothermal circulation model \ref{fig:alkDemoH}A). The extent of seafloor weathering
is determined quantitatively by the alkalinity of the hydrothermal fluid (Figure \ref{fig:alkDemoH}B) and the amount of carbonate rock produced.

\begin{figure}[h!]
\begin{center}
\includegraphics[scale=0.40]{Figs/alkDemoH2.eps}
\end{center}
\caption{Results from a preliminary experiment of hydrothermal circulation through and geochemical alteration of oceanic basalt. Initial and boundary conditions are diagrammed in Figure \ref{fig:model}. The model has not been completed so this output reflects experiments that do not include full coupling of the circulation and geochemical models (Section 4.1.3). A) The hydrothermal circulation model is in steady-state with respect to flow (shown as streamlines) and fluid temperature (shown as colors) at t = 384 years and t = 768 years. B) Alkalinity (shown as colors) varies across the two-dimensional domain as cations are leached more rapidly from warmer fluid-saturated basalt. Alteration continues through t = 768 years and the geochemical model is not in steady-state.}
\label{fig:alkDemoH}
\end{figure}

\subsubsection{Coupling the models}
\label{sec:coupling}

Our final task in developing the seafloor weathering model is to
couple the hydrothermal circulation and geochemical models. 
This is achieved by simulating the transport of reactive species 
in solution by advection and diffusion. In the
model, each grid cell is simulated as a parcel of incompressible
hydrothermal fluid that occupies volume fraction $\phi$ (porosity),
and a parcel of incompressible rock which occupies volume fraction $1
- \phi$. Solute concentration in an individual cell can change by diffusion, 
advection determined by hydrothermal circulation, and reactions that partition 
components between between the mobile and the immobile phases. As a 
result, the reactive transport equation includes, from left to right, a diffusion 
term, an advection term, and a reaction term

\begin{equation}
\frac{\partial c}{\partial t}  = -\frac{1}{\phi}[D \bigtriangledown \cdot (\phi \bigtriangledown c) - \bigtriangledown \cdot (c \textbf{q}) + \Sigma R_i],
\label{eq:adr}
\end{equation}

where $c$ is solute concentration, $\phi$ is porosity, $D$ is a
diffusion constant, and $R_i$ are the reactions that change the amount
of solute $c$.

We will simulate reactive transport of all dissolved species (Equation
(\ref{eq:adr})) using finite volume methods described by
\citet{leveque2004}, thereby coupling the hydrothermal circulation
simulation to the geochemical simulation. We will parallelize the code
using Message Passing Interface (MPI) so that it can be run on Abbot's
Beowulf cluster. The geochemical model is much more numerically
intensive than the hydrothermal circulation model, so we will use one
CPU to do the hydrothermal circulation calculations and distribute
each the geochemical calculations for each gridbox to a separate CPU.


\subsection{Constraining the effect of seafloor weathering on climate regulation}
\label{sec:constraining}

A main objective of this research is to explore seafloor weathering as
a direct climate feedback. The seafloor weathering model is a detailed
simulation of hydrothermal circulation, basalt alteration and
carbonate precipitation, which means there are many parameters
available for variation within realistic ranges. We will perform systematic
variation of model parameters from a base ``reference simulation'' in
order to investigate the possible sensitivities of carbon uptake
rate. 

\subsubsection{Direct Climatic Control on Seafloor Weathering}

The rate of basalt alteration has been demonstrated to depend on
temperature \citep{gislason1993, oelkers2001, gislason2003}. If
seafloor weathering temperatures are set by deep ocean temperatures,
which reflect the climate of high latitudes where ocean water cools
and sinks, a negative climate feedback could result. \citet{brady1997}
explored the case of seafloor weathering temperatures being controlled
by climate, but conceded that understanding the seafloor weathering
feedback strength requires a better picture of the thermal regime that
drives it. They suggested both deep ocean water and tectonic processes
could exert control over the rate of seafloor weathering.

To investigate the relative influences of ocean temperature and
crustal heat flux on CO$_2$ uptake, we will run a series of
experiments at a range of seawater temperatures (top boundary
condition) and crustal heat fluxes (bottom boundary condition). The
deep ocean temperature will vary in simulations from -1.0$^{\circ}$C
to 40.0$^{\circ}$C. We will test higher bottom water temperature than
are inferred from the $\delta$O$^{18}$ of benthic foraminifera during
the Cenozoic \citep{zachos2001} because the bottom water temperature
could have been much higher during the Archaen \citep[although there
are other explanations for the observed
$\delta$O$^{18}$,][]{kasting2006atmospheric} and on other planets
\citep{abbot2012}. These variations in deep ocean temperature could
reflect global mean temperature variations of a smaller magnitude, as
predicted by polar amplification in global climate models
\citep{masson-delmotte2005}.



We obtain the heat flux ($q$) lower boundary condition using a half-space
cooling model \citep{turcotte2001}
%
\begin{equation}
q(y,t) = \frac{k(T_m - T_o)}{\sqrt{\pi k t}} exp \left (\frac{-y^2}{4 k t} \right),
\label{halfSpace}
\end{equation}
%
where where $T_m$ is mantle temperature and $T_o$ is surface
temperature. To investigate the effect of variable crustal heat
export, we will vary the difference in temperature between the mantle
and ocean ($T_m - T_o$) from the present-day value of $T_m - T_o =
1300^{\circ}$C to $T_m - T_o = 1600^{\circ}$C, a high
estimate of Archean mantle temperatures \citep{bickle1986}. Although a
half-space cooling model of oceanic crust is a simplified model of
actual lithospheric cooling, it is an appropriate choice for this
applications for two reasons: 1) this model fails to correctly capture
the thickness of old oceanic crust ($>\sim $ 80 Ma) because it allows
for the creation of an infinitely thick plate, but low-temperature
hydrothermal processes cease at $\sim$65 Ma and 2) this model does not
take into account near-axis heat flux variations, like those predicted
by a more complex analytical thermal model that takes into account the
size of a magma chamber \citep{sleep1975}, but low-temperature
hydrothermal processes do not occur in near-axis systems where magma
chamber heat fluxes are important.

\subsubsection{Sediment Cover}

Low-temperature basalt alteration has been shown to be controlled in
part by sediment cover, which can restrict access of deep ocean water
to hydrothermal systems and confine chemical exchange to diffusion
through sediment \citep{wheat2000, alt1999, snelgrove1996}. We will
investigate the sensitivity of the seafloor weathering model's carbon
uptake rate to sediment cover by varying the thickness and composition
of sediment overlaying crustal basalt. Sediment thickness will range
from  the no sediment case to the extreme, but possible, 200 m
thick sediment cover case, as inferred by seismic data of the Juan De
Fuca ridge flank \citep{rohr1994}. The no-sediment and low-sediment
cases correspond to a water world planet or the early Earth, which
would not have continent-derived sediments \citep{eriksson2007,
  romans2013}. We will also explore the effect of composition and
permeability of a sediment cover by using two different sediment
compositions: a clay-rich column, with a computed effective
permeability of 4.0x10$^{-17}$ m$^2$ and a silt-rich sediment column,
with a computed effective permeability of 5.1x10$^{-15}$ m$^2$
\citep{snelgrove1996}.

\subsubsection{Dissolved Inorganic Carbon}

The formation of carbonates in low-temperature hydrothermal systems
depends on the availability of dissolved inorganic carbon (DIC) in the
deep ocean \citep{sleep2001, alt1999}. Variations in atmospheric
CO$_2$ throughout Earth's history have resulted in deep ocean DIC
fluctuations, as the atmosphere and oceans have maintained approximate
chemical equilibrium \citep{berner1983, grotzinger1993,
  berner2001}.  The Archean and early Paleoproterozoic
oceans are thought to have been more acidic and enriched in CO$_2$
\citep{grotzinger1993}. Although an upper limit of oceanic DIC
concentration has not been estimated, thermodynamic calculations by
\citet{shibuya2010} suggest hydrothermal fluids in high-temperature
basalt hosted hydrothermal systems had 0.2 -- 0.3 mol kgw$^{-1}$, in
contrast with the modern value of up to 0.02 mol kgw$^{-1}$. We will
vary the deep ocean DIC concentration from the modern value to the
estimate of high-temperature vent fluids to assess how carbonate
storage varies with deep ocean DIC concentration.

\subsubsection{River Potassium Flux}

Potassium is added to seawater from rivers and high-temperature
hydrothermal alteration fluids and is consumed in low-temperature
hydrothermal fields. Assuming a constant flux of potassium from
high-temperature hydrothermal alteration, an increase in potassium in
the ocean could reflect a change in continental weathering and might
represent an indirect effect of temperature variation. Because
potassium content has been relatively constant over the past billion
years, \citet{demicco2005} proposed that the ocean efficiently buffers
potassium by taking more into oceanic crust when available. One of
these possible buffering reactions, potassium feldspar formation,
releases alkalinity by the release of Ca$^{2+}$
\\

%\begin{equation}
\begin{small}
\ce{2NaAlSi_3O_8  + 7CaAl_2Si_2O_8 + 2K^+ + 8SiO_2 +12H^+ = 2KAlSi_3O_8 + 2NaAl_7Si_{11}O_{30}(OH)_6 + 7Ca^{2+}},
\end{small}
%\label{kspar}
%\end{equation}
\\

so a change in potassium river flux could affect the carbon uptake
rate of seafloor weathering by changing ocean alkalinity
\citep{coogan2013}. We will run a series of simulations
with varying potassium concentrations in seawater to assess the effect
on carbon uptake rate. We will vary potassium seawater concentration
from 40 ppm K to 1200 ppm K to reflect a range of 10\% of present-day
potassium concentration (399.1 ppm K) to tripling of present day
potassium concentration \citep{nordstrom1979}. A value of 10\% present
day potassium in the ocean would reflect no potassium from rivers and
all from high-temperature hydrothermal alteration. Because a change in
potassium from continental sources would likely be accompanied by
additional changes in major cations, we will vary potassium for
three separate values of calcium content, 206 ppm (50\% present day),
412 ppm (100\% present day), and 618 ppm (150\% present day).

\subsubsection{Ocean Basement Permeability}

Despite the significant effect that ocean basement permeability has on flow
of hydrothermal fluids, we have a limited understanding of
permeability within the upper ocean basement. A review of core-scale
measurements and borehole in situ measurements of basaltic oceanic
crust suggests a range of bulk permeabilities of 10$^{-22}$m$^2$ to
10$^{-13}$m$^2$ for upper oceanic basaltic crust, while indirect
measurements such as temperature and flow logs as well as numerical
models suggest a range from 10$^{-16}$m$^2$ to 10$^{-9}$m$^2$
\citep{fisher1998}. These ranges span several orders of magnitude
because permeability varies with depth by orders of magnitude. Bulk
permeability of ocean basement has been shown to have a profound
effect on hydrothermal flow patterns \citep{wang1996}, and we plan to
evaluate the sensitivity of the seafloor weathering model's carbon
uptake rate on bulk permeability of upper oceanic basement.

\subsubsection{Proposed Simulations}

Table \ref{tab:exp} contains a complete list of proposed simulations
for exploring the seafloor weathering model's sensitivity as described
in the previous sections. The parameter choices for the ``reference
simulation'' are, unless otherwise specified in Table \ref{tab:exp},
present day seawater composition by \citet{nordstrom1979}, ocean
bottom temperature of 2.1$^{\circ}$C, no sediment cover, crustal heat
flux according to a half-space cooling model with $T_m - To =
1300^{\circ}$C, permeable basement thickness of 500m inferred from
seismic measurements of the Juan de Fuca ridge flank \citep{rohr1994}
and initial basement permeability of 3.7x10$^{-14}$ m$^2$ for 5.9 Ma
crust \citep{anderson1982}. All simulations will be run assuming an
initial crustal age of 3 Myr until 65 Myr.

The focus of these simulations is to determine how climatic and
tectonic forcing control CO$_2$ uptake by oceanic crust. The relevant
output to be calculated by these simulations is total CO$_2$ uptake
per unit area by each experiment by the end and time series of CO$_2$
uptake. Additionally, assuming a constant CO$_2$ degassing flux, a
constant continental silicate weathering CO$_2$ consumption flux, and
the current distribution of oceanic crust ages, the rate of CO$_2$
uptake by seafloor weathering can be used to estimate pCO$_2$ for each
experiment, providing a quantitative assessment of seafloor
weathering's role as a climate buffer.

\begin{table}[h!]
\caption{Proposed numerical simulations for investigating sensitivity of seafloor weathering's carbon uptake rate on sediment properties.} 
\centering
\begin{tabular}{rrrrrrr} 

\hline \hline
\multicolumn{3}{c}{\cellcolor{grey!25}Varying ocean bottom temperature (Section 4.2.1)}\\
\hline  
Simulation \# & Ocean bottom temperature [$^{\circ}$C] & $T_m - T_o$ [$^{\circ}$C] \\ [0.5ex]   
\hline  
1 & -1.0 -- 40.0 & 1300 -- 1600 \\

\hline \hline
\multicolumn{3}{c}{\cellcolor{grey!25}Varying sediment properties (Section 4.2.2)}\\
\hline  
Simulation \# & Sediment thickness [m] & Sediment permeability [m$^{2}$] \\ [0.5ex]   
\hline  
3  & 0 -- 200 & 4.0x10$^{-17}$, 5.1x10$^{-15}$ \\

\hline \hline
\multicolumn{3}{c}{\cellcolor{grey!25}Varying deep ocean DIC concentration (Section 4.2.3)}\\
\hline  
Simulation \# & Deep ocean DIC concentration [mol kgw$^{-1}$] \\ [0.5ex]   
\hline  
2  & 0.02 - 0.30 \\

\hline \hline
\multicolumn{3}{c}{\cellcolor{grey!25}Varying seawater potassium content (Section 4.2.4)}\\
\hline  
Simulation \# & [K] [ppm] & [Ca] [ppm] \\ [0.5ex]   
\hline  
4 & 40 -- 1200 & 206 -- 618  \\

\hline \hline
\multicolumn{3}{c}{\cellcolor{grey!25}Varying bulk basement initial permeability (Section 4.2.5)}\\
\hline  
Simulation \# & log$_{10}$(permeability [m$^2$])  & \\ [0.5ex]   
\hline  
5 & -15 -- -10&  \\


\hline      
\end{tabular}
\label{tab:exp}
\end{table}


\subsection{Applying the model to the Archean carbon cycle}
\label{sec:paleoclimate}

Though continental silicate weathering appears to dominate CO$_2$
buffering on present-day Earth, \citet{sleep2001} and
\citet{zahnle2002} proposed that seafloor weathering was a significant
CO$_2$ buffer during early Earth history. While continental silicate
weathering draws CO$_2$ out of the atmosphere at a
temperature-dependent rate, they proposed seafloor weathering in warm
($\sim$20$^{\circ}$C) hydrothermal systems consumes CO$_2$ at a rate
dependent on pCO$_2$. During the Archean, greater mantle radioactive
heat production resulted in higher crustal heat fluxes
\citep{mckenzie1974, bickle1986} and more warm oceanic crust would
have been available for seafloor weathering. Additionally, continental
crust might not have been a widespread feature of Earth's surface
until the Late or Middle Archean \citep{zegers2001}, so
low-temperature seafloor weathering would have played a larger role in
buffering CO$_2$.

The Archean and early Paleoproterozoic oceans are thought to have been
more acidic and enriched in CO$_2$ \citep{grotzinger1993}. Although an
upper limit of oceanic DIC concentration has not been estimated,
thermodynamic calculations by \citet{shibuya2010} suggest hydrothermal
fluids in high-temperature basalt hosted hydrothermal systems had 0.2
-- 0.3 mol kgw$^{-1}$, in contrast with the modern value of 0.02 mol
kgw$^{-1}$. Additionally, ocean sulfate concentration has been
suggested to have been significantly lower than modern, resulting in
lower anhydrite formation in high-temperature hydrothermal systems
than that observed in the present day \citep{grotzinger1993,
  arvidson2006}.

%During the Archean, seawater was more acidic and enriched in CO$_2$ than modern [[grotzinger1993, ohmoto2004, rouchon2008]], which resulted in alkaline vent fluids [[shibuya2010]] and carbonate formation in high-temperature hydrothermal systems.

To investigate the relative influences of ocean temperature and
crustal heat flux on CO$_2$ uptake during the Archean, we will run a
series of simulations at a range of seawater temperatures with crustal
heat fluxes in accordance with predicted high Archean mantle
temperatures In these simulations, system parameters will differ from
those discussed in Section \ref{sec:constraining} to reflect Archean
conditions. Seawater compositions will be set according to predictions
of seawater compositions by fluid inclusion analysis
\citep{deronde1997}, thermodynamic calculations \citep{shibuya2010},
and Precambrian sedimentary rocks \citep{grotzinger1993}. The crustal
heat flux will be set according to a half-space cooling model with
$T_m - To = 1300^{\circ}$C -  $1600^{\circ}$C to to reflect mantle temperatures ranging
from the present day value to a high estimate of Archean mantle temperatures 
\citep{hoffman1988}. There will be no oceanic sediment cover to reflect early Earth 
conditions. Primary basalt 
composition will vary between present-day and tholeiitic (Mg- and Fe-rich) Archean
mid-ocean ridge basalt samples from Cleaverville, Western Australia (Aus.)
\citep{ohta1996} and West Greenland \citep{komiya2004} (GL). Deep ocean
temperatures will range from -1.0$^{\circ}$C to 40.0 $^{\circ}$C.

Like the model sensitivity experiments (Section \ref{sec:constraining}), these
experiments will focus on how CO$_2$ uptake is affected by climatic
forcing. The model output that we will focus on is the total CO$_2$
uptake per area of crust. Assuming constant CO$_2$ degassing and a
continent-free Archean planet, the CO$_2$ uptake rate of crust can
provide an estimate of pCO$_2$ for each experiment, providing a
quantitative assessment of seafloor weathering's role as a climate
buffer.

\begin{table}[h!]

\caption{Proposed numerical simulations for investigating seafloor weathering under Archean conditions.\\} 
\centering
\begin{tabular}{c|c|c|c|c} 

\hline  
\hline
 %\tiny &&&&\\[0pt]
 \multicolumn{5}{c}{\cellcolor{grey!25}Varying Archean conditions}\\
 \hline
Sulfate [mol kgw$^{-1}$]  & DIC [mol kgw$^{-1}$] & Ocean bottom T [$^{\circ}$C] & $T_m - T_o$ [$^{\circ}$C] & Basalt composition \\
%\tiny &&&&\\[0pt]
\hline
%\tiny &&&&\\[0pt]
0.0 -- 1.0 x 10$^{-4}$ & 0.002 -- 0.2 & -1.0 -- 40.0 & 1300 -- 1600  & Present-day, Aus., GL \\
%\tiny &&&&\\[0pt]
\hline      
\end{tabular}
\label{tab:archean}
\end{table}





\subsection{Broader Impacts}
\label{sec:outreach}

One of the most fundamental and important problems facing modern
society is climate change due to increased greenhouse gas levels, and
any work that addresses this issue \textit{benefits society}.  The
basic research into seafloor weathering proposed here brings us closer
to a comprehensive picture of the global carbon cycle, a framework
necessary for understanding the impact of industrial CO$_2$ on our
environment. Moreover, dealing with climate change will likely require
some combination of carbon sequestration and switching to alternative
energy sources, such as nuclear power. The proposed work addresses
these issues because the model we will develop can be easily modified
to be applied to carbon sequestration and nuclear waste disposal.

The primary purpose of this proposal is to obtain funding for Navah
Farahat to complete her PhD studies, which means that \textit{this
  proposal promotes teaching and training}. Since she is a woman, this
proposal also \textit{broadens participation of an under-represented
  group in the mathematical sciences}.

An important aspect of this proposal is our \textit{public
  outreach}. We will \textit{broaden dissemination to enhance
  scientific and technological understanding} by producing three
issues of an zine called ``Illustrated Topics in Geophysics,'' which
we will distribute for free on our websites and sell at cost at zine
stores in Chicago and Austin. The first issue will be ``The Carbon
Cycle,'' and the completed cover and first three pages are shown in
Figure~\ref{fig:illCover}. The zines will be designed to explain
important topics in Earth Science to the common person and will
contain vivid illustrations accompanied by a minimum of text.

We will also develop a website to distribute our model and findings to
other researchers and students. After we have completed the proposed
research, we will post the model code as well as documentation that
will explain how to run it. We will also set up a GUI that will allow
the user to play with a single box version of the geochemical model,
and thereby gain intuition into the relevant reactions. Finally, we
will post animations with sample output from a few simulations that
will show clearly how fluid enters hydrothermal systems, what it
reacts with, and what changes in the fluid and minerals result from
these reactions.

\begin{figure}[h!]
\begin{center}
\includegraphics[scale=0.053]{Figs/illPages0.pdf}
\end{center}
\caption{This figure is the completed cover and first three pages of of the first 
zine we will produce as part of the public outreach program of this proposal. 
The zines will be distributed online for free and in print at cost.}
\label{fig:illCover}
\end{figure}


\section{Project management and coordination between participants}
\label{sec:proj-management}

The three participants in this proposal, Navah Farahat, David Archer,
and Dorian Abbot, complement each other and make a balanced group that
brings together expertise on fluid dynamics (Abbot), geochemistry
(Archer), and advanced programming techniques (Farahat). Our combined
experience will allow us to develop the complex numerical 2D
reactive-transport model of seafloor weathering, test it, and apply it
to the study of climate. Farahat will take the lead in developing the
model and performing the simulations, while Archer and Abbot will
serve in an advisory role. We have been working closely for two years
performing the initial work for this project and will continue to do
so until we carry the work described in this proposal through to
completion. We meet weekly to discuss progress on the project and to
set goals for the next week. On a monthly basis we reevaluate our
larger-scale progress and make sure we remain on the right track.




\clearpage
\addcontentsline{toc}{section}{6\ \ \ References}
\addtocounter{section}{1}

\bibliographystyle{ametsoc}
\bibliography{biblio}{}
\end{document}
